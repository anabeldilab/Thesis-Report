En este apéndice, se detallará la configuración del entorno de desarrollo utilizado para este proyecto. El ecosistema de desarrollo se compone principalmente de Unity, el middleware ROS, y la librería Ros2ForUnity. Se explicarán los pasos necesarios para instalar y ajustar cada uno de estos componentes para asegurar un funcionamiento óptimo. Por último se explicará el funcionamiento y la experiencia tenida con otra librería llamada Unity-Robotics-Hub.


\section{NextMind SDK}

\subsection{Requerimientos del SDK}
Para poder usar el SDK de NextMind es importante tener en cuenta que tiene una serie de requerimientos que se muestran a continuación:

\subsubsection{Requisitos mínimos de hardware}
\begin{itemize}
  \item Soporte Bluetooth LE (4.0)
  \item Gráficos - DX9 shader model 2.5, equivalente a Intel HD 2500
  \item CPU - Intel i5-4590, equivalente a AMD FX 8350
  \item RAM - 8 GB
\end{itemize}

\subsubsection{Compatibilidad de software}
\begin{itemize}
  \item Última versión probada: Unity - 2022LTS
  \item Versión oficial soportada: Unity - 2020LTS, 2019LTS
  \item Plataformas - Windows 10 de 64 bits, Apple macOS de 64 bits (Mojave, Catalina, Big Sur)
  \item Software probado y aprobado - Oculus Rift, Oculus Quest 1 y 2, HTC Vive y Pro, HoloLens 1
  \item Compatibilidad de software verificada - Valve Index, HoloLens 2
\end{itemize}

\subsection{Instalación del SDK}

La instalación del NextMind SDK es un proceso sencillo que se realiza a través de la plataforma Unity. En primer lugar, es necesario descargar el SDK desde el sitio web oficial de NextMind o directamente desde el Unity Asset Store. Tras la descarga, el SDK puede ser importado al proyecto de Unity utilizando el menú ``Assets''.

\subsection{Desarrollo con el SDK}

Con el SDK correctamente instalado, ahora se puede comenzar a desarrollar aplicaciones que interactúan con el dispositivo NextMind. La documentación del SDK proporciona una serie de tutoriales y ejemplos que pueden ayudar a comenzar.