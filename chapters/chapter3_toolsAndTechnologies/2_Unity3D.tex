\section{Unity 3D}

Unity3D\footnote{Unity3D: \url{https://unity.com/}} es una de las plataformas de desarrollo de videojuegos 3D más importantes del mercado. Esta herramienta polifacética no solo se utiliza en la creación de videojuegos, sino también en la producción de simulaciones, experiencias de realidad virtual (VR) y aumentada (AR), y aplicaciones interactivas para diversas industrias, que podrían ser útiles para su implementación con NextMind.



La plataforma utiliza principalmente los lenguajes de programación C\# y UnityScript, una variante de JavaScript. Ofrece un entorno de desarrollo visual que permite a los creadores diseñar, personalizar y controlar los entornos virtuales y sus interacciones sin necesidad de programar cada detalle desde cero. Unity3D también incluye una extensa biblioteca de activos y herramientas predefinidas para acelerar el proceso de desarrollo.



En el contexto de este Trabajo de Fin de Grado, Unity3D adquiere una importancia especial por su integración con el dispositivo NextMind, siendo esta plataforma la única que el sistema de desarrollo de NextMind permite utilizar.
