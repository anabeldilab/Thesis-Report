\section{Contexto y justificación}

Para comenzar a desentrañar el fascinante mundo de la robótica combinada con las interfaces cerebro computador, es importante entender el significado de esto último. Una interfaz cerebro computador (BCI, por sus siglas en inglés) establece una comunicación directa entre la actividad eléctrica del cerebro y dispositivos externos, como ordenadores o dispositivos electromecánicos. Estas interfaces se clasifican en diferentes niveles de invasividad según la proximidad de los electrodos y el tejido cerebral, pudiendo ser no invasivas, parcialmente invasivas o invasivas.



La integración de este tipo de interfaces y dispositivos robóticos presenta numerosas ventajas y aplicaciones. Una de las ventajas más destacadas es su capacidad para ayudar a personas con problemas de movilidad a realizar tareas cotidianas, como el manejo de una silla de ruedas, el control de una prótesis o la interacción con sistemas domóticos.



Este Trabajo de Fin de Grado en Ingeniería Informática, se enfoca en el estudio e integración de una interfaz cerebro computador específica, denominada NextMind, para el control de dispositivos electromecánicos.



NextMind es una startup de neurotecnología que ganó el premio a la Mejor Innovación en CES 2020. El 8 de diciembre de 2020, sacó su kit de desarrollo para su dispositivo BCI llamado por el mismo nombre \cite{BusinessWire2020} Posteriormente, fue adquirida por Snap Inc, una reconocida empresa de tecnología y cámaras estadounidense\cite{SnapInc} Este acontecimiento ocasionó un gran cierre del producto, disminuyendo el nivel de soporte del kit de desarrollo. % aquí quiero escribir por qué a pesar de su cierre es buena idea seguir con él en esta investigación



NextMind destaca por su robustez en la detección  en la actividad cerebral de la respuesta ante determinados est\'imulos visuales, lo cual lo convierte en una herramienta confiable de cara al reto que supone un BCI. Además, su asequible precio lo hace accesible para investigadores y desarrolladores. Asimismo, su relativa comodidad de uso garantiza una experiencia satisfactoria para los usuarios durante sesiones prolongadas de interacción con el BCI.



Dada la creciente importancia de las interfaces cerebro-computadora en numerosas aplicaciones, el objetivo principal de este trabajo es investigar y desarrollar una aplicación en el que se use NextMind para el control de dispositivos electromecánicos, con el fin de explorar su viabilidad y potencial en este área.