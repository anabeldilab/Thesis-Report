\section{Estado del arte}

El uso de los BCI para el control de dispositivos está ganando popularidad debido a su potencial para mejorar la vida diaria de las personas con movilidad limitada. También aparecen nuevas ideas para su uso habitual, como por ejemplo, siendo parte de la interfaz de usuario en sistemas de realidad virtual (VR) o realidad aumentada (AR). Este proyecto se centra en los métodos que utilizan cascos EEG de tipo malla y, particularmente, en aquellos basados en SSVEP, dada su fiabilidad y alto rendimiento.
Específicamente, se utilizará el dispositivo NextMind, una BCI existente en el mercado que utiliza el enfoque SSVEP. La selección de este dispositivo se basa en su accesibilidad, robustez y comodidad.



Un estudio relevante es ``\textit{Robotic Arm with Brain}'', que utiliza una BCI no invasiva basada en EEG para controlar un brazo robótico \cite{RoboticArmWithBrain} Esta investigación proporciona un marco sobre cómo se pueden utilizar los BCIs para controlar dispositivos físicos de manera efectiva.



Por otra parte, el estudio ``\textit{A New SSVEP based BCI Application on the Mobile Robot in A Maze Game}'', se centra en el desarrollo de un juego de laberinto controlado por una BCI basada en SSVEP con el objetivo de mejorar la calidad de vida de las personas con enfermedad de las neuronas motoras. A la interfaz cerebro-computador se le proporcionan 4 opciones de movimiento:  ``girar en sentido antihorario'', ``girar en sentido horario'', ``desplazarse hacia adelante'' y ``desplazarse hacia atrás''. Para facilitar esas elecciones se utiliza un monitor LCD para mostrar iconos que parpadean a diferentes frecuencias. Esta técnica aprovecha la respuesta de las neuronas en el lóbulo occipital, encargado de procesar los estímulos visuales, que se sincronizan con la frecuencia de la luz percibida por los ojos. Los resultados demostraron que este control basado en SSVEP puede brindar entretenimiento a las personas con dicha enfermedad en el contexto del juego de laberinto.\cite{SSVEPBCIRobotMazeGame} La relación con el presente trabajo radica en el funcionamiento del BCI, que es similar en términos de los estímulos utilizados, así como en las opciones de movimiento proporcionadas.



Finalmente, el estudio ``\textit{A Telepresence Mobile Robot Controlled With a Noninvasive Brain–Computer Interface}'' presenta un sistema de telepresencia basado en EEG que permite a los usuarios tener presencia en entornos remotos a través de un robot móvil con acceso a Internet. El sistema utiliza una BCI basada en el potencial P300 y un robot móvil con capacidades de navegación autónoma y orientación de la cámara \cite{TelepresenceMobileRobotBCI} La relación con el trabajo de fin de grado se encuentra en el concepto de la telepresencia, donde se busca utilizar dispositivos controlados por una interfaz cerebro-computadora para lograr una presencia remota en entornos a través de un robot móvil. En este proyecto, se aplicarán ideas similares para permitir a los usuarios controlar objetos y tener una presencia remota mediante la interfaz cerebro-computadora.



La revisión de los estudios anteriores, especialmente aquellos relacionados con BCIs que utilizan el enfoque SSVEP, ha sido muy útil. En particular, el dispositivo NextMind, que se utiliza en este proyecto, ha proporcionado un marco útil para el diseño y la implementación del estudio.