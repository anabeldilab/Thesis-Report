\section{Antecedentes y estado actual del sistema}

\subsection{Antecedentes}
Los sistemas BCI nacen de la necesidad de añadir otro tipo de interacción más, aparte de los que ya estamos acostumbrados con el fin de dar soluciones diferentes y eficaces. 
La idea de las BCIs se originaron junto con el descubrimiento de la actividad eléctrica en el cerebro humano y el desarrollo del electroencefalograma (EEG), que fue el primer test que registró la actividad cerebral e identificó señales oscilatorias.
Hoy en día existen numerosas aplicaciones para las interfaces cerebro-computador. En el caso en el que se centra este proyecto es en los BCIs no invasivos. Dentro de los BCIs no invasivos hay diferentes formas de recolectar la información del cerebro. La gran mayoría del trabajo sobre BCI publicado involucra BCI no invasivas basadas en EEG. Usando este método, hay interfaces que ponen la atención en diferentes zonas del cerebro según su finalidad.

\begin{itemize}
  \item BCIs basados en EEG motor imagery[1] (MI-BCI), siendo útil por ejemplo, en el ámbito de la medicina, desarrollado para la rehabilitación motora en personas que han sufrido algún accidente cerebrovascular, causando parálisis de las extremidades superiores o inferiores. Se suelen usar exoesqueletos robóticos o la estimulación eléctrica de los músculos[2]. 
  \item BCIs basados en EEG P300[3], brinda información sobre los procesos cognitivos en el cerebro, como la memoria, la atención, la concentración y la velocidad del procesamiento mental. 
  \item BCIs basados en Steady-State Visual Evoked Potential, SSVEP, actualmente es el enfoque más popular, proporcionando un alto rendimiento y una comunicación confiable para ser una BCI no invasiva[4].
  \item BCIs basados en múltiples métodos, como por ejemplo, una BCI que funciona con inteligencia artificial cognitiva que es capaz de procesar lenguaje natural y la forma en la que se expresan los humanos, para enviar las señales neuronales, procesarlas y transmitirlas a un altavoz con inteligencia artificial. Haciendo que haya personas que puedan volver a comunicarse con el hecho de pensar las palabras y un asistente por voz las reproduciría[5].



\end{itemize}

\subsection{Estado actual del sistema}



