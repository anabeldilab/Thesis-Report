\section{Conclusion}
Early disease diagnosis is a pending task in today's medicine and, due to deep learning, specialists can get help from a trained neural network to fulfill their objective and prevent illnesses. This way artificial intelligence adds a new field in which it can influence enormously in the coming years and, with this kind of applications people can start seeing it as a tool and not as threat to their jobs. Referring specifically to glaucoma, early detection is really tough considering there are no precise or specific synthoms, that is why extremely experienced doctors are needed for this job. With deep learning models this job gets highly simplified, as long as there is a correctly labeled and of a significant amount set of images.
\par
This web application supplies specialists the possibility of, uploading a retinography, obtain an automatic diagnosis about if that eyeball has glaucoma. Everything works thanks to a convolutional neural network, working as a service and to which calls are made every time an image is uploaded. These also allow the neural network to continue improving it's diagnoses, taking into account it receives a correct feedback about the judgement from the expert.
\par
The main improvements done to this web application let the customer have a better user experience. With the inclusion of users, the system becomes more custom to each one, and adding that to the analized images history implemented, the doctor is now able to check already finished diagnoses, next to  the relative data, in addition to being able to compare them all one to another to obtain overall better results. Furthermore, the new metadata introduced referring to the image and to the patient, allows the expert to identify all the finished diagnoses, thus include new comments on the relevant reports. All of this is possible due to the relational database, that grants us with the possibility to connect in a better way different entities. The update done to some frameworks boosts our performance and reduces the risks of a leak in some components. Lastly, administrators are now able to manage the main data saved in the database, on top of having the ability to export the images and its data from the database in a fast and easy process to a directory.
\par
The greatest problem I encountered during this project was having to use and understand a system with very little documentation and developed by a person I did not have any kind of touch. Therefore, having a complete idea of all the code that was written and starting up the system were really time consuming, more than I firstly expected. Moreover, the code had no tests and that made me test manually any change I did. Moreover, fitting myselve to the coding style that was already in the project made it more difficult for me in the beginning.
\par
Despite all os the complications already stated, and that it was my first time taking part of a fullstack project like this one, it made me understand in a better way about Python's functionning and utilties as a language used in most of the neural network activities and deep learning models, along with learning about artificial intelligence and its wide range of possibilities. On top of that, it gave me the possibility to discover about tiped components in React as a result of using it on top of React, and the operations that can be done with add-ons of this language. Finally, using docker in the backend helped me understand containerization and the way to connect containers between them, I had read and learned about this technique previously, nevertheless only theoretically, and in this project I could put this knowledge to practice.
\par
\section{Future activities.}
Including the possibility for the user to include options about the image processing and creating a table in the database referring to explanatory images of the diagnosis opens a future development path. This will grant that new automatic learning models can be studied and provide users the chance to get an explication about the diagnosis thanks to saliency maps. Moreover, I have come up with some ideas to improve the application in the future. Giving the user the capability of managing their images, deleting those they will not need anymore or modifying the relative data would be a meaningful improvement that is not very time consuming. On top of that, it would be excellent for users to send images and diagnoses to other experts within the application and get their opinion, whenever they have any doubts about the result.
\par
Finally, on a longer run, the system could have other neural networks implemented for diagnosing other eye diseases, and, thanks to this, enlarge remarkably the services of the application: taking advantage of the images that are already taken and uploaded to the system, it could produce an evaluation report of that image regarding all the relevant eye diseases. 
\par