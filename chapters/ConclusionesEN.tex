\section{Conclusions}

In this section, we will discuss the conclusions.

Firstly, I have to highlight that all objectives have been met, although not everything went as expected.

\subsection{Prototype development}
As pointed out in chapter \ref{chapter:challengesanddifficulties}, the need to learn ROS from scratch represented a significant commitment of time and autonomous effort for understanding and assimilating information. Additionally, I had no previous experience programming in Unity, which also implied a learning process from scratch. Regarding specific programming on the ESP32, although it was unknown territory, I benefited from previous knowledge acquired in a degree course called Embedded Systems, which provided a solid foundation for general programming in microcontrollers. This was particularly useful during the programming of the pan-tilt from the microcontroller with respect to PWM, a task that, although it required some research, was simpler thanks to the skills acquired in that subject. Of all aspects of the project, what presented the least complications was the operation of the Neurotags and the NextMind system itself.

In terms of my personal perception of the tools used, I found ROS fascinating and it is a tool I would like to continue using in the future. My experience with Unity, on the other hand, was not as positive due to its logging and error management system, which I consider complicated to locate faults. As for ESP-IDF, it provided me with a satisfactory experience, perceiving it as a more modular structure than a micro-ROS project for Freertos microcontroller programming.

\subsection{Interpretation of data and conclusions obtained}
 
The results of the study point to the effectiveness and adaptability of NextMind in different environments and situations. As a BCI, it is accessible and resilient, not being influenced by aspects such as the presence of hair. Its efficient performance, both indoors and outdoors, and its resistance to variations in ambient light, open up new possibilities for future advances in BCI devices. It was observed that precise calibration and prior experience with BCIs improve its efficiency, underlining the need for familiarization with the technology. As for comfort, most users rated NextMind positively, indicating its potential for extended use.

\section{Future lines}

\subsection{Prototype improvements}

With a view to continuing the development and improvement of the project, various possible actions and modifications have been identified to improve its operation and expand its utility.

\begin{itemize}

\item \textbf{Improve the camera}: The camera's resolution could be improved to more clearly capture the QR codes. A more powerful camera would contribute to improving the effectiveness and precision in the detection of the codes.

\item \textbf{Implement an undo state function}: It would be useful to incorporate a function in the prototype to quickly reverse the state of the camera within the QR detection, for example, through a button or a key on the keyboard that allows removing a state in a simple way.

\item \textbf{Improvements in calibration and adjustment of the device}: According to the opinions collected, the device in general is not difficult to put on, but some people with certain types of hair may have difficulties. Therefore, options could be explored to facilitate its adjustment. In addition, it could be beneficial to have an alternative method for calibration in case of failure.

\item \textbf{Incorporate acoustic feedback}: Some users, apparently, were so focused on the stimulus that they did not notice the visual feedback provided by the NeuroTag. To facilitate the user's perception, an additional acoustic stimulus could be added to the visual one.

\item \textbf{Expand controls with NeuroButtons}: It would be convenient to implement numbered NeuroButtons from 1 to 5, where each number represents a specific amount of automatic movements that the Pan-Tilt system can perform.

\item \textbf{Explore more devices}: The application of the BCI can be expanded to other devices, such as a robotic chair or a robot, which would open new possibilities for remote control of various devices through the brain-computer interface.

\end{itemize}


\subsection{Research improvements}

\begin{itemize}
\item \textbf{Expand the sample}: A larger sample will allow for more representative and reliable results in tests, and more accurate conclusions can be drawn.

\item \textbf{Explore the impact of ambient noise}: Considering that this prototype is designed for outdoor use, it would be interesting to study how difficulty concentrating in noisy environments can affect the performance of NextMind. This could involve conducting more tests in these contexts.

\item \textbf{Investigate the effect of ADHD}: Studying the effects of Attention Deficit Hyperactivity Disorder (ADHD) on the use of NextMind can provide useful insights as concentration is vital in this BCI.

\item \textbf{Study the adaptation to the device in people with visual conditions}: 
During the tests, certain patterns were found, although not enough, between visual conditions and score in calibration. The study of how people with different visual conditions (myopia, astigmatism, hypermetropia, etc.) adapt to the use of NextMind could be interesting.

\item \textbf{Analyze the effect of prolonged training}:

 Investigate if the effectiveness of the device can improve with practice and prolonged training and determine the time needed to observe significant improvements. As during the tests, clear indications were seen that having previous experience in the use of BCI resulted in better mastery of it.

\item \textbf{Study the effect of fatigue}: Investigate how fatigue or mental tiredness can affect the performance of NextMind. This could help determine how long it can be effectively used before fatigue begins to affect performance.

\end{itemize}